\section{Data Description}

Obtaining the data used in this paper is challenging, especially when an Electronic Health Records (EHR) dataset is involved \cite{wang2021EHRa}. The data acquisition involves multiple complicated steps. The first step is to obtain both geospaital data and EHR data that are compatible for projection. The second step is to pre-process the data for removing empty and erroneous values. The final step is to transform the data into a format that is suitable for visualization.

\subsection{Geospatial Data}

Geospatial data, commonly known as shapefiles, were obtained from the following sources. After acquiring these data, we used QGIS \cite{qgisWelcome} to manually adjust projections and merge them into one shapefile. Finally, mapshaper \cite{blochMapshaper} is used to converted the merged shapefile into a TopoJSON \cite{TopoJSON} file to reduce the file size in order to improve the performance of our software.

\subsubsection{Clinical Commissioning Groups}

Clinical Commissioning Groups (CCGs) are the main geographic unit of the National Health Service (NHS) in the UK \cite{nhsNHS}. The number of CCGs is changing each year due to the NHS annual re-organization, the most up-to-date shapefile is available from the Open Geography portalx \cite{opengeographyportalxOpen}. We decided to use the CCG shapefile from 2020 as at the time of writing, there is no EHR data published based on the latest CCG re-organization took place in April 2021.

\subsubsection{Rivers}

We used OpenStreetMap \cite{openstreetmapRelation} as our data source to obtain shapefiles for River Thames, River Trent and River Great Ouse in England. These rivers were chosen as they pass through regions with clusters of CCGs, and provide informative geographical and topological cues. 

We first obtained a relation ID by searching a river, e.g. River Thames, on OpenStreetMap. The obtained relation ID was used to construct a query (See Listing~\ref{overpass}) which enabled use to download the entire river shapefile through Overpass Turbo \cite{overpassturboOverpass}.

\begin{lstlisting}[caption={The query that downloads the shapefile of River Thames from OpenStreetMap via Overpass Turbo API.}, label={overpass},captionpos=b]
    relation(2263653);>>;
    out skel;
\end{lstlisting}

\subsection{EHR Data}

We obtained the Clinical Commissioning Group Outcomes Indicator Set (CCG OIS) from NHS Digital \cite{nhsdigitalClinical}. The OIS is a set of indicators that are used to measure the quality of care and the associated health outcomes in the NHS. Some datasets include:
\begin{itemize}
    \item Under 75 mortality
    \begin{multicols}{2}
        \begin{itemize}
            \item Cardiovascular disease
            \item Respiratory disease
            \item Liver disease
            \item Cancer
        \end{itemize}
    \end{multicols}
    \item Emergency hospital admission
        \begin{itemize}
            \item Stoke
            \item Alcohol-specific admission and readmission
            \item Coronary heart disease
            \item Readmissions within 30 days of discharge
            \item Children with lower respiratory tract infections
        \end{itemize}
\end{itemize}

For all datasets, a spreadsheet including the following data is provided:

\begin{itemize}
    \item Reporting period: Calendar year of registration
    \item Period of coverage: Start and end date or reporting period
    \item Breakdown: Organization type
    \item ONS code: UK Office for National Statistics CCG code
    \item Level: CCG Code
    \item Level description: CCG Name
    \item Gender
    \item Indicator value: Directly standardized mortality rate
    \item CI lower: lower 95\% confidence interval
    \item CI upper: upper 95\% confidence interval
    \item Denominator: The count of registered patients
    \item Numerator: Number of deaths
\end{itemize}