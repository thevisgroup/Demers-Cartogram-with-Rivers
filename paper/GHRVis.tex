\documentclass[Afour,sagev,times]{sagej}

\usepackage{moreverb,url}

\usepackage{float,hhline,array,tabulary,tabularx,etoolbox,multirow, multicol}

\newcommand{\algorithmautorefname}{Algorithm}

\usepackage[table, dvipsnames]{xcolor}

\definecolor{Mycolor1}{HTML}{4C5866} %#4C5866
\definecolor{Mycolor2}{HTML}{A6B6D2} %#A6B6D2
\definecolor{Mycolor3}{HTML}{67daff} %#67daff
\definecolor{Mycolor4}{HTML}{F7F7FE} %#F7F7FE
\definecolor{Mycolor5}{HTML}{666ad1} %#666ad1
\definecolor{Mycolor6}{HTML}{ffbb93} %#ffbb93
\newcommand{\mycolorbox}[2]{\begingroup\setlength{\fboxsep}{2pt}\colorbox{#1}{#2}\endgroup}

\newcommand{\bobgraph}[1]{\noindent\textbf{#1}}

% cite with author names
\newcommand{\citea}[1]{\citeauthor{#1}~\cite{#1}}

% draw a colored square
% color, width, height
\newcommand\coloredsquare[3][black]{\textcolor{#1}{\rule{#2}{#3}}} 

% thumbnail image
\usepackage{wrapfig}
\newcommand{\thumbnail}[2]{
    \begin{samepage}

      \setlength{\intextsep}{3pt}
      \setlength{\columnsep}{3pt}
      \begin{wrapfigure}{r}{0.4\columnwidth}
        \vspace{-15pt}
        \begin{center}
          \includegraphics[width=0.38\columnwidth]{figure/related/"#1".png}
        \end{center}
      \end{wrapfigure}
      #2
      
    \end{samepage}
}

% \newcommand{\thumbnail}[2]{
%   \setlength{\intextsep}{2pt}
%   \setlength{\columnsep}{2pt}

%   \noindent\begin{minipage}[t]{0.4\columnwidth}
%     \vspace{-10pt}
%     \begin{center}
%       \includegraphics[width=0.98\columnwidth]{figure/related/#1.png}
%     \end{center}
%   \end{minipage}%
%   \begin{minipage}[t]{0.6\columnwidth}
%     \noindent\leftskip=0pt  #2
%   \end{minipage}
% }


\usepackage{algpseudocode}
\usepackage{algorithm}
% switch for algorithmic
% New definitions
\algnewcommand\algorithmicswitch{\textbf{switch}}
\algnewcommand\algorithmiccase{\textbf{case}}
\algnewcommand\algorithmicdefault{\textbf{default}}

\algdef{SE}[SWITCH]{Switch}{EndSwitch}[1]{\algorithmicswitch\ #1\textbf{:}}{\algorithmicend\ \algorithmicswitch}%
\algtext*{EndSwitch}%

\algdef{SE}[CASE]{Case}{EndCase}[1]{\algorithmiccase\ #1\textbf{:}}{\algorithmicend\ \algorithmiccase}%
\algtext*{EndCase}%

\algdef{SE}[DEFUALT]{Default}{EndDefault}{\algorithmicdefault\textbf{:}}{\algorithmicend\ \algorithmicdefault}%
\algtext*{EndDefault}%

\algnewcommand\algorithmicforeach{\textbf{for each}}
\algdef{S}[FOR]{ForEach}[1]{\algorithmicforeach\ #1\ \algorithmicdo}

% code listing
\usepackage{listings}
\lstdefinestyle{customc}{
  belowcaptionskip=1\baselineskip,
  breaklines=true,
  xleftmargin=\parindent,
  language=C,
  showstringspaces=false,
  basicstyle=\footnotesize\ttfamily,
  keywordstyle=\bfseries\color{green!40!black},
  commentstyle=\itshape\color{purple!40!black},
  identifierstyle=\color{blue},
  stringstyle=\color{orange},
}

\lstdefinestyle{customasm}{
  belowcaptionskip=1\baselineskip,
  xleftmargin=\parindent,
  language=[x86masm]Assembler,
  basicstyle=\footnotesize\ttfamily,
  commentstyle=\itshape\color{purple!40!black},
}

\lstset{escapechar=@,style=customc}

% list with styles
\usepackage{enumitem}
% UpdateNodePosition

\newcommand{\nodeList}{\mathbcal{L}}
\newcommand{\nodeListEach}{l^i}
\newcommand{\nodeListVPSC}{\mathbcal{L}_{vpsc}}
\newcommand{\nodeListVPSCEach}{l^i_{vpsc}}
\newcommand{\nodeSize}{s}
\newcommand{\nodeSizeMax}{s_{max}}
\newcommand{\stalemate}{t}
\newcommand{\crossRiver}{cross}

% move position

\newcommand{\node}{\bm{n}}
\newcommand{\nodeVPSC}{\bm{n}_{vpsc}}

% check river crossing

\newcommand{\riverEdgeList}{\bm{R}}
\newcommand{\riverEdgeListEach}{r^i}
\newcommand{\nodeBoundingBox}{b_n}
\newcommand{\riverEdgeBoundingBoxEach}{b_{\riverEdgeListEach}}
\newcommand{\boundingBoxInt}{b_{int.}}
\newcommand{\nodeEdge}{e_n}
\newcommand{\edgeBoxInt}{e_{int.}}

% derive corridor
\newcommand{\CorridorLength}{c_l}
\newcommand{\CorridorWidth}{c_w}
\newcommand{\PointP}{p}
\newcommand{\Edge}{e}
\newcommand{\EdgeParallel}{e_p}
\newcommand{\EdgeParallelA}{e_{p^1}}
\newcommand{\EdgeParallelB}{e_{p^2}}
\newcommand{\Distance}{d}
\newcommand{\PointStart}{p_s}
\newcommand{\PointEnd}{p_e}
\newcommand{\nodeInCorridorEach}{{n}_{in}^i}
\newcommand{\nodeInCorridorEachDx}{\Delta x_{\nodeInCorridorEach}}
\newcommand{\nodeInCorridorEachDy}{\Delta y_{\nodeInCorridorEach}}


% derive corridor point
\newcommand{\Scale}{scale}
\newcommand{\dx}{\Delta x}
\newcommand{\dy}{\Delta y}
\newcommand{\Vector}[2]{\protect\overrightarrow{#1#2\strut}}


\usepackage[colorlinks,citecolor=red,urlcolor=red]{hyperref}

\newcommand\BibTeX{{\rmfamily B\kern-.05em \textsc{i\kern-.025em b}\kern-.08em
T\kern-.1667em\lower.7ex\hbox{E}\kern-.125emX}}

\usepackage[nameinlink]{cleveref}

\setcounter{secnumdepth}{3}

\begin{document}

\title{Demers Cartogram with Rivers}


% \includegraphics[width=\textwidth,keepaspectratio]{figure/cover.png}
%         \captionof{figure}{The River Thames passing through a cartographic representation of the NHS CCGs in London and surrounding areas.}

\begin{abstract}
    Cartograms serve as representations of geographical and abstract data, employing a value-by-area mapping technique. A relative of the Dorling cartogram, the Demers cartogram, utilizing squares instead of circles to represent regions. This alternative approach allows for a more intuitive comparison of regions, utilizing screen space more efficiently. However, a drawback of the Dorling cartogram and its variants lies in the potential displacement of regions from their original positions, ultimately compromising legibility, readability, and accuracy. To tackle this limitation, we propose a novel hybrid cartogram layout algorithm that incorporates dynamic topological elements, such as rivers, into Demers cartograms. The presence of rivers significantly impacts both the layout and visual appearance of the cartograms. Through a user study conducted on an Electronic Health Records (EHR) dataset, we evaluate the efficacy of the proposed hybrid layout algorithm. The obtained results illustrate that this approach successfully retains key aspects of the original cartogram while enhancing legibility, readability, and overall accuracy.
\end{abstract}

\keywords{Cartogram, Demers Cartogram}

\maketitle

\section{Introduction and Motivation}

Cartograms are representations of geographical and abstract data based on a value-by-area mapping combining statistical and geographical information \cite{dent2009Cartography,inoue2011New}. 
Various styles of cartograms have been proposed and implemented for applications such as urban planning \cite{harris2018Mapping, arranz-lopez2021Enduser}, natural hazard forecasting \cite{pappenberger2019Cartograms, park2020Flood}, conservation and environmental planning \cite{galluzzi2018Mapping, rocchini2019Cartogramming}, political and social demographics \cite{breitzman2018Using, alieva2021How}, and decision-making for public health \cite{gao2020Visualising, sack2021Visualizing}.

Among the four types of cartogram categorized in a survey by \citea{nusrat2016State} (See \Cref{sec:{Related Work}} for definitions of contiguous, non-contiguous, rectangular, and Dorling), a trade-off is made between types of accuracy (See \Cref{table:accuracy}).
For this project, we focus on non-contiguous cartograms like Demers cartograms because they facilitate statistical comparison between regions, they can make good use of screen space, and comparison of regions is useful when studying Electronic Health Records (EHR) data.
Demers cartograms offer the advantage in cases where the data is not directly correlated to region sizes. Also, the comparison of magnitudes becomes an area estimation task, which is more effective for a numeric data encoding \cite{munzner2014Visualization}.
Building on Demers cartograms \cite{ian2002Cartogram}, we introduce and develop novel features, such as rivers, aiming to improve the readability and geographical accuracy without sacrificing statistical accuracy.
Standard Demers cartograms are composed of square nodes. As such, this can reduce their legibility.
We implement a new hybrid cartographic layout algorithm that combines rivers with the placement of the nodes representing geo-referenced regions. 
We hypothesize that introducing rivers improves the overall legibility of a cartogram. 
By \textit{legibility} we mean readability and ability to interpret the cartogram.  
To assess this hypothesis, we designed an experimental setup where participants engaged in correspondence and location tasks as part of a user study.
To reduce error and make efficient use of screen space, the algorithm also updates the position of rivers to accommodate the node layout.
We then apply the algorithm to a real-world case study using EHR data to evaluate the result.
We present a user study that demonstrates its effectiveness.

Our contributions include:

\begin{itemize}
    \item A new variant of Demers cartograms that incorporates rivers to improve readability and recognizability,
    \item A novel hybrid layout algorithm that combines node positions with features such as rivers,
    \item A user study evaluation of the technique with an application to EHRs.
\end{itemize}

The results of the user study indicate that rivers can improve the legibility of cartograms.
One of the major challenges involved is developing a layout algorithm that handles different shapes.
In other words, the hybrid layout algorithm is novel because it handles different types of elements: square representing regions and polylines representing rivers.
Another challenge we overcome in developing the algorithm is to resolve stalemate situations introduced by rivers while minimizing error.

\section{Related Work}

Cartograms combine statistical and geographical information in thematic maps. We follow the categorization of cartograms by \citea{vankreveld2004Rectangular}: contiguous, non-contiguous, rectangular, and Dorling. \citea{nusrat2016State} summarize three major accuracy dimensions for cartograms: statistical, geographical, and topological. Each cartogram design may make accuracy trade-offs between dimensions, we provide an overview in Table \ref{table:accuracy}.

{
\renewcommand{\arraystretch}{1.5}
\begin{table}[!b]
	\centering
	\resizebox{\columnwidth}{!}{
		\begin{tabulary}{\columnwidth}{|*{5}{l|}}
			\hhline{~|*{3}{-}|~}
			\multicolumn{1}{c|}{} & \multicolumn{3}{c|}{\cellcolor{Mycolor2} \textbf{Accuracy}} \\
			\hhline{~|*{4}{-}}
			\multicolumn{1}{c|}{\textbf{Catogram Type}} &
			\textbf{Statistical} &
			\textbf{Geographical} &
			\textbf{Topological} &
			\textbf{Contiguity} \\
			\hline
			Contiguous & \cellcolor{Mycolor3}Variable & \cellcolor{Mycolor3}Variable & Accurate & Yes \\
			\hline
			Non-contiguous & \cellcolor{Mycolor5}\textcolor{white}{Accurate} & \cellcolor{Mycolor5}\textcolor{white}{Shape is accurate} &\cellcolor{Mycolor6}Inaccurate & No   \\
			\hline
			Rectangular & \cellcolor{Mycolor3}Variable & Shape is inaccurate & \cellcolor{Mycolor3}Variable & Yes   \\
			\hline
			Dorling & Accurate & Inaccurate & Inaccurate & No  \\
			\hline
		\end{tabulary}
	}
	\caption{\textcolor{Mycolor3}{\textbf{Trade-off between dimensions}}. \textcolor{Mycolor6}{\textbf{Dimension sacrificed}} in order to improve \textcolor{Mycolor5}{\textbf{target dimension}}'s accuracy.}
	\label{table:accuracy}
\end{table}
}


Dorling cartograms are non-contiguous and do not preserve geography and topology. A Dorling cartogram is statistically accurate, regions are represented by circles and the statistic of interest is represented by the circle size \cite{dorling2011Area}. 
In Demers Cartograms, a variant of Dorling, squares are used instead to capture a certain level of topology \cite{cano2015Mosaic}. Dorling cartograms are unable to maintain topological accuracy as circles are often repositioned to remove overlaps. 

Rectangular cartograms are contiguous and do not preserve geographical accuracy \cite{raisz1934Rectangular}. Depending on the variant, a rectangular cartogram may trade between statistical and topological accuracy.

Mosaic cartograms are contiguous and sacrifice statistical accuracy to preserve some level of geographical accuracy \cite{cano2015Mosaic}. Some variants are able to preserve topological accuracy as well.


\citea{warf2008Geography} use a Dorling cartogram to represent the religious diversity in the United States. \citea{sun2010Effectiveness} visualize 1996 US election data and 2005 China population date using Dorling, Mosaic, and contiguous cartograms. \citea{cruz2017Adapted} adapts a Dorling cartogram with both contiguous and non-contiguous cartograms to represent the gender pay gap in Portugal. \citea{gao2020Visualising} present a Dorling cartogram to illustrate COVID-19 infections in China. \citea{tong2018cartograms} use a Dorling cartogram to visualize health-related date by regions in England. \citea{nusrat2020Recognition} investigate the memorability of contiguous and Dorling cartograms by using multiple datasets including demographics, Agriculture, and Retail data in the US. See Table \ref{table:region vs node} for a list of literature that adopts cartograms for visualization with corresponding geographical regions and node counts.


{
\renewcommand{\arraystretch}{1.5}
\begin{table*}[!tb]
	\centering
	\resizebox{\textwidth}{!}{
		\begin{tabulary}{\textwidth}{|*{4}{l|}r|c|}
			\hhline{~|*{4}{-}|~}
			\multicolumn{1}{c|}{\textbf{Literature}} &
			\textbf{Title} &
			\textbf{Cartogram Type(s)} &
			\textbf{Geographic Region(s)} &
			\textbf{Number of Nodes} &
			\multicolumn{1}{c}{\textbf{Year}} \\
			\hline
			
			% \citea{auber2007Geographical} & \citetitlea{auber2007Geographical} & US & 19 & 2007 \\
			% \hline
			\citea{warf2008Geography} & \citetitlea{warf2008Geography} & Dorling & US & 3,142 & \citeyear{warf2008Geography} \\
			\hline
			\citea{sun2010Effectiveness} & \citetitlea{sun2010Effectiveness} & Dorling, Mosaic, Contiguous & US, China & 34 - 49 & \citeyear{sun2010Effectiveness} \\
			\hline
			\citea{cruz2017Adapted} & \citetitlea{cruz2017Adapted} & Dorling, Non-contiguous, Contiguous & Portugal & 2,882 & \citeyear{cruz2017Adapted} \\
			\hline
			\citea{tong2018cartograms} & \citetitlea{tong2018cartograms} & Demers & England & 209 & \citeyear{tong2018cartograms} \\
			\hline
			\citea{gao2020Visualising} & \citetitlea{gao2020Visualising} & Dorling & China & 34 & \citeyear{gao2020Visualising} \\
			\hline
			\citea{nusrat2020Recognition} & \citetitlea{nusrat2020Recognition} & Contiguous, Dorling & Portugal & 49 & \citeyear{nusrat2020Recognition} \\
			\hline
			
			% \multicolumn{1}{c|}{\textbf{Total unique papers: 51}} & 24&15&12& \multicolumn{1}{c}{}  \\
			% \hhline{~|*{3}{-}|~}

		\end{tabulary}
	}
	\caption{}
	\label{table:region vs node}
\end{table*}
}

\section{Data Description}

 {
  \begin{figure}[tb!]
      \centering
      \includegraphics[width=0.6\columnwidth]{figure/ccg.png}
      \caption{A map of 135 CCGs in England as of 2020, obtained from the Open Geography Portalx \cite{opengeographyportalxOpen} with EPSG:4326 (WGS84 - World Geodetic System) as the Coordinate Reference System (CRS).}
      \label{fig:ccg}
  \end{figure}
 }

Obtaining the heterogenous data can be challenging, especially when an EHR dataset is involved \cite{wang2021EHRa}. The first step is to obtain both geospatial boundaries and EHR data. The second step is to pre-process the EHR data to remove empty and erroneous values. The final step is to transform the data into a format that is suitable for cartograms. Geospatial boundaries, or shapefiles, were obtained from the sources described here.

\bobgraph{Choropleth Shapefile:} Clinical Commissioning Groups (CCGs) are the primary administrative and geographic unit of the National Health Service (NHS) in the UK \cite{nhsNHS}. The number of CCGs changes over time due to NHS re-organization. The most up-to-date shapefile is available from the Open Geography Portalx \cite{opengeographyportalxOpen}. We decided to use the CCG shapefile from 2020 at the time of writing, due to the absence of public EHR data published based on the latest CCG re-organizations that took place in 2021 and 2022.

\bobgraph{River Shapefiles:} We used OpenStreetMap \cite{openstreetmapRelation} as our data source to obtain shapefiles for River Thames, River Trent, and River Great Ouse in England. These rivers were chosen as they are well-known rivers and pass through regions with dense populations, and provide informative geographical and topological cues. Including additional and smaller rivers remains future work.

We first obtain a relation ID by searching for a river, e.g. River Thames, on OpenStreetMap. The relation ID is used to construct a query (See Listing~\ref{overpass}) which enables the user to download the entire river shapefile using Overpass Turbo \cite{overpassturboOverpass}.

\begin{lstlisting}[float=tp,caption={The query that downloads the shapefile of River Thames from OpenStreetMap via the Overpass Turbo API.}, label={overpass},captionpos=b]
    relation(2263653);>>;
    // River Great Ouse: 2798097
    // River Trent: 2863468
    out skel;
\end{lstlisting}

After acquiring the shapefiles, we used QGIS \cite{qgisWelcome} to manually adjust projections, and convert them into GeoJSON files. Finally, mapshaper \cite{blochMapshaper} is used to merge and convert the GeoJSON files into a TopoJSON \cite{TopoJSON} file. TopoJSON eliminates redundant coordinates in the data, improving the rendering speed of our implementation. See \Cref{table:pre-processing_result} on page \pageref{table:pre-processing_result} for the pre-processing result. We describe the above steps in more detail in \Cref{app:pre-processing}.

\bobgraph{EHR Data: }We obtained the Clinical Commissioning Group Outcomes Indicator Set (CCG OIS) from NHS Digital \cite{nhsdigitalClinical}. The OIS is a set of indicators that are used to measure the quality of care and the associated health outcomes in the NHS. Some datasets include:
\begin{itemize}
    \item Under 75 mortality: cardiovascular disease, respiratory disease, liver disease, and cancer
    \item Emergency hospital admission: stroke, alcohol-specific admission and readmission, coronary heart disease, re-admissions within 30 days of discharge, children with lower respiratory tract infections
\end{itemize}

For all datasets, a spreadsheet including the following is provided:

\begin{itemize}
    \item Reporting period: Calendar year of registration
    \item Period of coverage: Start and end date or reporting period
    \item Breakdown: Organization type
    \item ONS code: UK Office for National Statistics CCG code
    \item Level: CCG Code
    \item Level description: CCG Name
    \item Gender
    \item Indicator value: Directly standardized mortality rate
    \item CI lower: lower 95\% confidence interval
    \item CI upper: upper 95\% confidence interval
    \item Denominator: The count of registered patients
    \item Numerator: Number of deaths
\end{itemize}

Each CCG has a unique ONS code, which is used to link the CCG shapefile with the statistical data.

\section{\software}

\algoref{alg:UpdateNodePosition} describes the overview of \software. We first load all processed data from \secref{sec:{Data Description}} and apply VPSC \cite{dwyer2006fast} recursively to remove overlaps. We chose VPSC over other node overlap removal algorithms since VPSC is able to provide spread minimization and node movement minimization while maintaining a good level of global shape preservation \cite{chen2020Node}. During each recursion, we check node trajectories (See \algoref{alg:check river crossing}) and move back nodes that crossed a river. A recursion ends when 1) no overlaps are present; 2) no nodes crossed a river. We then increase the node size and repeat the recursion until the entire map canvas is filled.

% \begin{noindent}

\begin{algorithm}[tb!]
    \caption{The procedure to update node positions by removing overlaps and prevent nodes from crossing rivers.}\label{alg:UpdateNodePosition}
    \textbf{Variables:} \\
    $NodeList \gets$ a list of nodes representing CCGs \\
    $Node_{size}\gets$ the current size of all nodes \\
    $Node_{inc} \gets$ the increment node size of each iteration \\
    $Node_{max} \gets$ the maximum size of a node \\
    $Node_{stale} \gets$ the maximum number of iterations before a stalemate \\
    \begin{algorithmic}[1]
        \Procedure{UpdateNodePosition}{}
        \While{$ Node_{size} < Node_{max} $}

        \State $ nodeList_{ora} \gets $ RunVPSC ($ NodeList $)

            \For{$ node^i_{ora} \in nodeList_{ora} $}

                \If {MoveNode ($ Node^i, node^i_{ora} $)}
                    \State $ cross \gets $ \Call{CheckRiverCrossing}{$ Node^i, node^i_{ora} $}

                    \If{$ cross = True $}
                        \State $ Node^i.stale \gets Node^i.stale + 1 $

                        \If{$ Node^i.stale < Node_{stale} $}
                            \State \Call{MoveNode}{$ Node^i $}
                        \Else
                            \State \Call{DeriveCorridor}{$ Node^i, node^i_{ora} $}
                        \EndIf

                    \EndIf

                \EndIf

            \EndFor

        \EndWhile

        \State $Node_{size} \gets Node_{size} + Node_{inc}$

        \EndProcedure
    \end{algorithmic}
\end{algorithm}


\begin{algorithm}[tb!]
    \caption{The procedure to update a node's position.}\label{alg:move position}
    \begin{algorithmic}[1]
        \Procedure{MoveNode}{$ Node, Node_{new} $}
        \If{$ Node.x = Node_{new}.x $ and $ Node.y = Node_{new}.y $}
            \State \Return{$ False $}
        \EndIf
        
        \State init. $ x,y $

        \If{$ Node_{new}$ is $ undefined $}
            \State $previous \gets Node.history.pop() $
            \State $ x \gets previous.x,~y \gets previous.y $
        \Else
            \State $ x \gets Node_{new}.x,~y \gets Node_{new}.y $
            \State $ Node.history.push(x, y) $
        \EndIf

        \State $ Node.x \gets x $
        \State $ Node.y \gets y $
        \State \Return{$ True $}
        \EndProcedure
    \end{algorithmic}
\end{algorithm}

%\end{noindent}

\subsection{River Cross Checking}

We use rivers as topological boundaries to prevent nodes from crossing them. When a node's position changes, we check if the node's trajectory intersects any segment of a river, see \algoref{alg:check river crossing}. A boundary collision detection is performed to reduce the number of line intersection detections required.


% \begin{noindent}
\begin{algorithm}[tb!]
    \caption{The procedure to check if a node crosses a river.}\label{alg:check river crossing}
    \textbf{Variables:} \\
    $RiverEdges \gets$ a list of river edges formed by two points \\

    \begin{algorithmic}[1]
        \Procedure{CheckRiverCrossing}{$ Node, Node_{ora} $}
            \For{$ edge \in RiverEdges $}
                \State $ node_b \gets $ GenerateBoundary ($ Node, Node_{ora} $)
                \State $ edge_b \gets $ GenerateBoundary ($ edge $)
                \State $ collide \gets $ DetectBoundaryCollision ($ node_b, edge_b $)
                \If{$ collide = True $}
                    \State $ node_l \gets $ GenerateLine ($ Node, Node_{ora} $)
                    \State $ edge_l \gets $ GenerateLine ($ edge $)
                    \State $ intersect \gets $ DetectLineIntersection ($ node_l, edge_l $)
                    \If{$ intersect = True $}
                        \State \Return{$ True $}
                    \EndIf
                \EndIf
            \EndFor
            \State \Return{$ False $}
        \EndProcedure
    \end{algorithmic}
\end{algorithm}
%\end{noindent}

\subsection{Corridor Derivation}

As the VPSC always tries to produce an optimal node layout where node spread and movement are minimized, a node might be repeatedly traveling between two positions due to the lack of available space, creating a stalemate situation for recursions, as shown in \figref{fig:stalemate}. In order to create space for VPSC to generate a new layout, we introduce a user-adjustable parameter, $ Node_{stale} $, to break the stalemate: if a node is traveling between two positions for more than $ Node_{stale} $ iterations, it is considered to be in a stalemate. A stalemate triggers a corridor derivation which moves all nodes within the corridor by $ \mathcal{D} $ set by the user (See \figref{fig:corridor} and \algoref{alg:derive corridor}).

{
\begin{figure}[tb!]
    \centering
    \includegraphics[width=\columnwidth]{figure/stalemate.png}
    \caption{A stalemate situation is when a node repeatedly travels between two positions (A and B) for more than $Node_{stale}$ iterations. }
    \label{fig:stalemate}
\end{figure}
}

{
\begin{figure}[tb!]
    \centering
    \includegraphics[width=\columnwidth]{figure/corridor.png}
    \caption{When a stalemate occurs, a corridor (black rectangular) is derived based on the current (A) and previous (B) positions of the node that crossed a river, as described in \algoref{alg:derive corridor}. All nodes within the corridor are moved by $ \mathcal{D} $ in the direction of $ \vec{BA} $. }
    \label{fig:corridor}
\end{figure}
}

% \begin{noindent}

\begin{algorithm}[tb!]
    \caption{The procedure to derive a corridor to move included nodes. We use a SVG canvas, where the point of origin (0,0) is located at the top left corner, with the x-axis extending to the right and the y-axis extending to the bottom, there is no negative axes.}\label{alg:derive corridor}
    \textbf{Variables:} \\
    $\mathcal{C}_l \gets$ the length of a corridor \\
    $\mathcal{C}_w \gets$ the width of a corridor \\
    $\mathcal{D} \gets$ the distance to move nodes within the corridor \\
    $Node_{size}\gets$ the current size of all nodes \\
    \begin{algorithmic}[1]
        \Procedure{DeriveCorridor}{$ Node, Node_{old} $}
        \State MoveNode ($ Node $)

        \State $ \Delta x \gets Node.x - Node_{old}.x,~ \Delta y \gets Node.y - Node_{old}.y $

        \State $ slope \gets \frac{\Delta x}{\Delta y}$

        \State $ pLine = $ \Call{DerivePoint}{$ \frac{-1}{ slope } $, $ Node $, $ \mathcal{C}_w $}
        \State $ sideLine_1 = $ \Call{DerivePoint}{$ slope $, $ pLine.a $, $ \mathcal{C}_l $, $ True $}
        \State $ sideLine_2 = $ \Call{DerivePoint}{$ slope $, $ pLine.b $, $ \mathcal{C}_l $, $ True $}
        \State $ corridor \gets
            \begin{bmatrix}
                pLine.a &
                pLine.b \\

                sideLine_1.a &
                sideLine_2.b \\
            \end{bmatrix} $

        \For{$ node^i $ inside $ corridor $}
            \State $ destination = $ \Call{DerivePoint}{$ \frac{-1}{ slope } $, $ node^i $, $ \mathcal{D} $}

            \State MoveNode ($ node^i, destination.b $)
        \EndFor

        \EndProcedure

        \item[]

        \Procedure{DerivePoint}{$ Slope, Point, Length, IsSideLine $}
        
        \State init. $ a $, $ b $
        \Switch{$Slope$}
            \Case{$0$}
                \If{$ IsSideLine = True $}
                    \State $ a.x \gets Point.x $
                \Else
                    \State $ a.x \gets Point.x + Length $
                \EndIf
                \State $ a.y \gets Point.y $
                \State $ b.x \gets Point.x - Length $,~$ b.y \gets Point.y $
            \EndCase
            \Case{$ \infty $ or $ -\infty $}
                \State $ a.x \gets Point.x $

                \If{$ IsSideLine = True $}
                    \State $ a.y \gets Point.y $
                \Else
                    \State $ a.y \gets Point.y + Length $
                \EndIf

                \State $ b.x \gets Point.x $,~$ b.y \gets Point.y - Length $
            \EndCase
            \Default
                \State $ dx = \sqrt{\frac{Length}{{1+Slope^2}}}$,~ $ dy = Slope \cdot dx $
                \If{$ IsSideLine = True $}
                    \State $ a.x \gets Point.x $,~$ a.y \gets Point.y $
                \Else
                    \State $ a.x \gets Point.x + dx $,~$ a.y \gets Point.y + dy $
                \EndIf

                \State $ b.x \gets Point.x - dx $,~$ b.y \gets Point.y - dy $
            \EndDefault

        \EndSwitch

        \State \Return{$ a, b $}

        \EndProcedure

    \end{algorithmic}
\end{algorithm}

%\end{noindent}

\color{blue}

\section{Evaluation}

\subsection{Hypotheses Formulation}

We formulate the following hypotheses to evaluate the effectiveness of our approach:

\textbf{H1:} The introduction of dynamic topological features can effectively improve the recognizability of CCGs.

\textbf{H2:} \dots

\subsection{Participants}

We recruited X participants, [add demographic analysis here].

\subsection{Datasets}

We used the following datasets for our evaluation:

\begin{itemize}
    \item Under 75 mortality from cardiovascular disease
    \item Emergency admissions for alcohol-related liver disease
\end{itemize}

135 CGGs are projected on the screen with a choropleth, we then generated another view with our visual design. The color in both visualizations are mapped to population. See \autoref{fig:task}.

{
    \begin{figure}[tb!]
        \centering
        \includegraphics[width=\columnwidth,keepaspectratio]{figure/evaluation/task.png}
        \caption{A typical task for participants. The left shows the choropleth map, and the right shows the cartogram. Both visualizations show the three longest rivers in England, and the color is mapped to CCG population. The target CGG will be blinking on the choropleth (shown in black), participants are asked to identify this CCG on the cartogram.}
        \label{fig:task}
    \end{figure}
}

\subsection{User Experiment}

We conducted a user experiment to evaluate the effectiveness of our approach. The experiment was designed to be conducted remotely, and participants were asked to complete the tasks on their own computers. The experiment includes four parts:

\textbf{P1:} The participants were asked read instructions and trainings provided in both text and videos. The instructions were designed to help participants understand the concepts used in the tasks. Videos are available at \url{https://www.youtu.be/playlist?list=PLL7sHvxLtD75fMtrUQrAdddjt3wfFkcWz}.

\textbf{P2:} The participants were given three practice tasks to familiarize themselves with the experiment. A demonstration of the task is also included in the instructional video.

\textbf{P3:} The participants were asked to complete 16 location tasks. The response and reaction time are recorded. These 16 CCGs were carefully selected to avoid extreme cases (thus biases the result), in terms of size, color, and location.

\textbf{P4:} The participants were asked to complete a questionnaire that consists of Likert Scale questions and open-ended questions.

The user experiment is available at \url{https://osf.io/q39w7}.

\color{black}

\section{Limitations and Future Work}
\label{sec:{Limitations and Future Work}}

In this section, we discuss some limitations of our work and future research directions.
Because this is a new concept, it opens the door for many future research directions.
The first limitation is the color map that we use to depict the data in our cartograms. 
We use D3's built-in interpolateRdYlGn color map, a diverging color scheme of red, yellow, and green.
However, we believe that the choice of color map can have a significant impact on the legibility of cartograms.
In the user study, we carefully avoid extreme values where the location or color of the CCG makes it easier to locate the target.
See \Cref{fig:extreme} for an example.
We plan to explore the impact of different color maps on the legibility of cartograms in future work.

{
    \begin{figure}[tb!]
        \centering
        \includegraphics[width=0.7\columnwidth,keepaspectratio]{figure/limitations/extreme.png}
        \caption{Due to color and relative location, we believe the CCGs in the black circle are easier to locate.}
        \label{fig:extreme}
    \end{figure}
}

Another limitation is the overlap removal algorithm (FNOR) we use.
We believe that developing a new algorithm with built-in constraint support can significantly reduce the time required to generate cartograms with rivers. Currently, the runtime of our layout algorithm is approximately 30 milliseconds for each iteration.

Future work also includes generalizations and extensions of the algorithm, e.g., the use of other features in the cartogram layout such as additional rivers, major highways, lakes, and coastlines, etc.
We also consider whether increasing the length of the rivers as the size of the nodes increases would be a useful option.
We would like to explore the case of river-river intersections and testing out more geographies such as the U.S. and Europe, which have more complex rivers.
We also considered the idea of deforming the rivers as part of the layout algorithm, however, this idea is open to future work.
Furture work also includes adding rivers to dorling-style cartograms.


\section{Conclusions}

In this paper, we present a novel approach to generate cartograms with rivers. We first propose a new algorithm to generate cartograms with rivers, and then we present a prototype to support the exploration of cartograms with rivers. We evaluate our approach using a user study, and the results partially support our hypotheses: the introduction of rivers improves the legibility but doesn't reduce the time needed to locate a region. We also discuss the limitations of our work and the future directions.



% \bibliographystyle{mslapa}
\urlstyle{same}
\bibliographystyle{plainnat}
\bibliography{references}


\cleardoublepage

\appendix

\section{Reproject Coordinate Reference System (CRS) of Shapefiles}
Shapefiles obtained use different CRS for projections. The river shapefiles use EPSG:4326 (WGS-84), whereas the CCG shapefile uses EPSG:27700 (OSGB36 - British National Grid). We first load all three river shapefiles into QGIS, followed by the CCG shapefile. QGIS automatically prompts to transform the latter into WGS-84, see \figref{fig:reproject}. The reprojection result is shown in \figref{fig:all_rivers}.

{
\begin{figure}[h!]
    \centering
    \includegraphics[width=\columnwidth]{figure/qgis/reproject.png}
    \caption{QGIS interface. A window prompt to transform the CCG shapefile's CRS into WGS-84.}
    \label{fig:reproject}
\end{figure}

\begin{figure}[h!]
    \centering
    \includegraphics[width=\columnwidth]{figure/qgis/all_rivers.png}
    \caption{The reprojection result produced by \figref{fig:reproject}, which includes the River Trent (red), River Great Ouse (green), and River Thames (orange).}
    \label{fig:all_rivers}
\end{figure}
}

\section{Procedure: DerivePoint}

% \begin{noindent}

    \begin{algorithm}[h!]
        \caption{Procedure to derive a point based an edge and a distance.}\label{alg:derive corridor point}
        \textbf{Input:} \\
        $ \Edge \gets $ the edge used to derive the new point \\
        $ \Distance \gets $ the distance between $ \PointP $ and $ \EdgeStart $ \\

        \textbf{Output:} \\
        A point, $ \PointP $, that is distance $ \Distance $ away from $ \EdgeStart $. \\
    
        \textbf{Local variables:} \\
        $ \dx, \dy \gets $ the differences in $ x, y $ for $ \EdgeStart $ and $ \EdgeEnd $ \\
    
        \begin{algorithmic}[1]
            \Procedure{DerivePoint}{$ \Edge $, $ \Distance $}
                \State $ \dx \gets \EdgeStart.x - \EdgeEnd.x $
    
                \State $ \dy \gets \EdgeStart.y - \EdgeEnd.y $
    
                \State $ \PointP.x \gets \frac{\dx}{\sqrt{\dx^2 + \dy^2}} \cdot \Distance $
    
                \State $ \PointP.y \gets \frac{\dy}{\sqrt{\dx^2 + \dy^2}} \cdot \Distance $
    
            \State \Return{$ \PointP $}
    
            \EndProcedure
    
        \end{algorithmic}
    \end{algorithm}
    
%\end{noindent}

\section{Procedure: DeriveParallelEdge}

% \begin{noindent}

\begin{algorithm}[h!]
    \caption{Procedure to derive an edge, $ \EdgeParallel $, that is parallel to $ \Edge $ with a distance of $ \Distance $.}\label{alg:derive corridor edge}

    \textbf{Input:} \\
    $ \Edge \gets $ the edge used to derive the parallel edge $ \EdgeParallel $ \\
    $ \Distance \gets $ the shortest distance between $ \Edge $ and $ \EdgeParallel $ \\

    \textbf{Output:} \\
    An edge, $ \EdgeParallel $, that is parallel to $ \Edge $ with a distance of $ \Distance $. \\

    \textbf{Local variables:} \\
    $ \dx, \dy \gets $ the differences in $ x, y $ for $ \EdgeStart $ and $ \EdgeEnd $ \\
    $ \Scale \gets $ the scale of $ \frac{\Distance}{\sqrt{\dx^2 + \dy^2}} $ \\

    \begin{algorithmic}[1]
        \Procedure{DeriveParallelEdge}{$ \Edge $, $ \Distance $}
            \State $ \dx \gets \EdgeStart.x - \EdgeEnd.x $

            \State $ \dy \gets \EdgeStart.y - \EdgeEnd.y $

            \State $ \Scale \gets \frac{\Distance}{\sqrt{\dx^2 + \dy^2}} $

            \State $ \EdgeParallel.start.x \gets \Scale \cdot -\dy + \EdgeStart.x $

            \State $ \EdgeParallel.start.y \gets \Scale \cdot \dx + \EdgeStart.y $

            \State $ \EdgeParallel.end.x \gets \Scale \cdot -\dy + \EdgeEnd.x $

            \State $ \EdgeParallel.end.y \gets \Scale \cdot \dx + \EdgeEnd.y $

        \State \Return{$ \EdgeParallel $}

        \EndProcedure

    \end{algorithmic}
\end{algorithm}

%\end{noindent}

\end{document}
